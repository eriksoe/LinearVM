\documentclass[a4paper]{book}

% Type constructors:
\newcommand\tyImport{{\ensuremath\textit{import}}}
\newcommand\tyVar{{\ensuremath\textit{var}}}
\newcommand\tyApp{{\ensuremath\textit{app}}}
\newcommand\tyBless{{\ensuremath\textit{bless}}}
\newcommand\tyCurse{{\ensuremath\textit{curse}}}
\newcommand\tyPtr{{\ensuremath\textit{ptr}}}
\newcommand\tyPad{{\ensuremath\textit{pad}}}
\newcommand\tyRecord{{\ensuremath\textit{record}}}
\newcommand\tyVariant{{\ensuremath\textit{variant}}}

\newcommand\tyVoid{{\ensuremath\textit{void}}}

\begin{document}

\tableofcontents
\part{Architecture}

\section{Introduction}

\subsection{Background}
% * The gap - platforms lacking proper concurrency guarantees
%   vs. BEAM which is concurrency-first but pays perhaps too much
%   - poor in destructive updates / ephemeral data structures.
% * C++ and "safe" patterns (esp. regarding smart pointers);
%   "A virtual machine for C++"
% * Library/API documentation (, state of): names & types
%   but always not lifecycle and threading constraints.


\subsection{What is LinearVM?}
% An interface - in the form of a program representation
% and well-formedness criteria providing certain guarantees.

\section{Design philosophy}

\subsection{Priorities}

The design priorities are, in order from most important to less important:

\begin{itemize}
\item Safety (type and concurrency)
\item Flexibility
\item Simplicity
\item Performance (cheapness in space and time).
\end{itemize}

\subsubsection*{Safety}
Safety is paramount. The only (lamentable) exception is the
necessary support for using native libraries, i.e. having a Foreign
Function Interface (FFI). Ideally, development tools should provide
good support for stress testing FFI bindings.

\subsubsection*{Flexibility}
It should be easy to model a wide variety of languages and paradigms on the
virtual machine -- to the extent, of course, that the features to be modelled
are type and concurrency safe.
Furthermore, since it's still early days and there's little
experience with using linear types to implement real systems,
we have no desire to fix decisions prematurely.

\subsubsection*{Simplicity}
Quoth Dijkstra, ``Simplicity is prerequisite for reliability''.
Quoth Sessions paraphrasing Einstein: ``Everything should be as
simple as it can be but not simpler''.
Given the constraints set down by safety and flexibility, we strive
for simplicity. This is a fine balance.

One core source of simplicity is following the ``provide mechanism
rather than policy'' design rule: Including the general principles in
the VM, but leaving specific details to be added on top of it later.

\subsubsection*{Performance}
Speed and compactness are not primary goals. We believe that with the
right basis, and suitably strong code invariants, fast execution can
come later. (Even Javascript has been pushed far these days.)

If the above design priorities mean that program details get more explicit
(e.g. involve more steps),
then so be it -- at least as long as it's no worse than a constant factor.
(This overhead is akin to that of proof-carrying code [TODO: link]).


\subsection{Still early days/Core business}
As mentioned, there's little experience with linearly typed systems in
practice.

A theme which may become recurrent is that a design decision will be
resolved with ``we don't know enough about the real-world uses of this
yet -- so we'll effectively delay this decision by providing a suitably
general mechanism.''

Once some experience has been gained, and the real-world uses are
better understood, such design decisions may be revisited.


\section{Requirements}
The virtual machine should have a type system which
enforces type safety and safety against data races.
%Deadlock prevention may also be added, although that is less certain
%at this point.

The type system should be enforced by a bytecode verifier.

In order to be of practical use, the VM should have a FFI (foreign
function interface).  This is of course in itself a compromisation of
type integrity and so forth, but it appears to be a practical necessity
-- all VMs
in practical use have it.  The linear type system would presumably
support FFI better than most other mechanisms, given that linear types are good at
describing management of resources such as entities provided by
foreign libraries.

\section{Design basics}
% All values are linear at the outset
% I.e. copying & disposal are explicit
% Consequences for exceptions

In the following, we sketch the most immediate consequences of the
requirements and priorities.

Design decisions tend to interact.
The following is a list of choices, approximately ordered by scope and
implications (wider and more basic decisions first).

\subsection{Linear types}
Firstly, supporting the combination of data race safety and imperative
paradigms leads to the use of linear types.

Modelling certain things in a concurrency-safe way becomes a great deal simpler
once you use linear types.
Choosing to support on linear types, and to rely on them for enforcing
certain invariants, is the basic design decision which makes LinearVM
stand apart.

Design decisions often interact with each other; this basic design choice
influences most other aspects of the VM design.

But should all values be regarded as linear, or only certain types? Numbers,
for instance, can be copied and destroyed without any special action.

We decide to treat all values as linear, for the following reasons:
\begin{itemize}
\item Non-linear types can easily be built on top of linear ones, by adding
  the trivial ``copy'' and ``dispose'' operations.
\item While number types which fit into the underlying machine's registers
  can be copied and destroyed without any special action, because they
  don't involve any external resources, the same is not true of
  e.g. ``bigints'' which involve some allocated memory.
  Thus, some number types are more linear-ish than others.
\item By treating all values as linear, we'll learn faster how to cope with
  different patterns of access and manipulation of linear values, and
  make them work for as wide a range of circumstances as the case allows.
\item Finally, there is simplicity in not having to deal with two
  kinds of values.
\end{itemize}

Thus, not only do we choose to support linear types -- we make them the
norm; in essence, all types are linear, and copying and destroying of values
is only possible if explicitly allowed.

\subsection{Layering}
The bytecode-verification algorithm is central to this project.
For security and robustness, it should be made as simple as the
typing rules allow -- yet flexible enough, of course, for real-world
programs.

What is least likely to change? The type system and typing rules, the
set of primitive types, or the set of operations on values?

The type system and its typing rules are presumably the most stable of
these three -- the most timeless part. Both the set of concrete types
and the set operations may well evolve over time.

Therefore, the type system and its verification algorithm is regarded
as the most fundamental component. Further up on the stack are the set
of types and further up yet the set of operations. These are less
fundamental to the degree that they're not actually fixed by the core
LinearVM design.

This serves two purposes which can be characterised as simplicity and flexibility:
Firstly, it simplifies the verifier, in that it does not need to incorporate
knowledge of all the primitve types and all the operations on them.
Secondly, it gives flexibility: the set of primitive types can be adjusted
more easily.
This design choice follows the separation design principle of ``mechanism,
not policy'' which reflects that it is beneficial to arrange software so
that stable, general parts are lower in the stack than parts which are
more specific and more prone to change.

Note that this architecture, in which the verification algorithm is
more or less ignorant about the concrete types and operations,
is in contrast to e.g. the JVM and CLR designs, in which the
byte-code verifier has hard-coded knowledge of a set of primitive types
and certain operations on them.

\subsection{Exceptions}

Linear types and exceptions don't play well together.
For standard exception propagation to work, one must assume that
all stack frames -- and therefore all values -- can be disposed of at will.
That is not the case, however, for linear values.

One might think up complex approximations to exception propagation.
In LinearVM, however, we take the simple solution:\footnote{
At least for the present. If experience shows that there are patterns to how exceptions are handled, then that might be supported more directly at a later point.}
At the VM level, \emph{exceptions do not exist} -- although they can be added on
a higher level of abstraction.
(This is quite like, at the level of physical processors, there is no
``exception'' concept.)

Operations which can either complete succesfully or terminate exceptionally
will be modelled using functions with \emph{multiple exits}.

\subsection{Multiple function exits}

% TODO

\part{Type System}

% Type constructors
% Type properties
% Typing rules
% Verifier
% Motivational typing examples:
% - bignums
% - concurrent queues
% - ref-counted objects
% - locks
% (- sparse matrix...)

\part{Types and Operations}

\chapter{Introduction}
\chapter{Small constructed types}

\section{Void}
``Void'' (\texttt{linearvm.basis.void}) is the unit type.
It is defined as the record type with no fields:
\begin{center}
  \verb|record {}|
\end{center}
The official name is just an alias.

Void is special in two respects:
\begin{enumerate}
\item it can be represented in zero bits;
  it is therefore not necessary to allocate space for void values at
  runtime, which means that they are in a sense ``free''.
  For this reason, void and derivatives (by $\tyBless$ and
  $\tyCurse$) are at times used to represent proofs and obligations.
\item it is trivial to construct and dispose of;
  it is therefore used (in conjunction with $\tyPad$) to represent e.g.
  uninitialized or emptied fields.
\end{enumerate}

Note that void is not usually used as a return type: there is a
difference between returning void (a record value with zero fields)
and returning zero values.

\section{Never}
``Never'' (\texttt{linearvm.basis.never}) is the bottom type.
It is defined as the variant type with no flavours:
\begin{center}
  \verb|variant {}|
\end{center}
The official name is just an alias.

Never is special in that a value of this type can not be constructed.
Function calls which have never as their first return value for some return
path are treated specially by the verifier: it assumes that the return
path in question will never be taken, and disregards that return edge
in the control-flow graph.
This is used in connection with the ``goto'' (\texttt{linearvm.basis.goto()})
pseudo-instruction.

\section{Boolean}
``Boolean'' (\texttt{linearvm.basis.boolean}) is the type of truth values.
It is defined as
\begin{center}
  \verb|variant {false: void, true: void}|
\end{center}
\subsection*{Operations}

\begin{itemize}
\item \texttt{and : (boolean,boolean) $\to$ boolean }
\item \texttt{or : (boolean,boolean) $\to$ boolean }
\item \texttt{not : (boolean) $\to$ boolean }
\item \texttt{if : (boolean) $\to$ void $|$ void }
  \\ This is equivalent to an ordinary ``case''
  (\texttt{linearvm.variant.case}) on the variant type.
\end{itemize}


% Void
% Bottom/never
% Boolean
\chapter{Integer types}
\chapter{Floating-point types}
\chapter{Bytestrings}
\chapter{Strings}
\chapter{Arrays}
\chapter{Records}
% create, fetch, store, borrow, deconstruct

\chapter{Variants}
\chapter{VTables}
% call_<F> : (vtable(\{\ldots, f: (a_1, \ldots a_n) -> o_1 \| \ldots \| o_m),
%            a_1, \ldots a_n) -> o_1 \| \ldots \| o_m.
% dispose: vtable(\ldots) -> ()
\chapter{Existentials}
% create : (T) -> (\tyExist(v,T)) when ...?

\chapter{Pointers}
% store: (T, ptr(pad(void, T))) -> ptr(T)
% fetch: ptr(T) -> (T, ptr(pad(void, T))) %% TODO: How to make work for different kinds of pointer?
% - ptr(V, T) ?

\chapter{Memory allocation}

% malloc: () -> bless(mallocated, ptr(pad(void, T)))
% free: bless(mallocated, ptr(pad(void, T))) -> ()


\part{File Format}
\part{Implementation}



\end{document}
