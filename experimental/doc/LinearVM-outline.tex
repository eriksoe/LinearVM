\documentclass[a4paper]{book}

\usepackage[plain]{fancyref}

% Type constructors:
\newcommand\tyImport{{\ensuremath\textit{import}}}
\newcommand\tyVar{{\ensuremath\textit{var}}}
\newcommand\tyApp{{\ensuremath\textit{app}}}
\newcommand\tyBorrow{{\ensuremath\textit{borrow}}}
\newcommand\tyBless{{\ensuremath{\textnormal{\textit{bless}}}}}
\newcommand\tyCurse{{\ensuremath\textit{curse}}}
\newcommand\tyPtr{{\ensuremath\textit{ptr}}}
\newcommand\tyPad{{\ensuremath\textit{pad}}}
\newcommand\tyRecord{{\ensuremath\textit{record}}}
\newcommand\tyVariant{{\ensuremath\textit{variant}}}

\newcommand\tyVoid{{\ensuremath\textit{void}}}

\newcommand\TypeDefinition[1]{\begin{quote}$\mathit{#1}$\end{quote}}
\newcommand\FuncSignature[2]{\begin{quote}$\mathit{#1}\ :\ \mathit{#2}$\end{quote}}
\newcommand\FuncSignatureII[3]{\begin{quote}$\mathit{#1}\ :\ \mathit{#2} \to$\\${}\quad \mathit{#3}$ \end{quote}}
\newcommand\TagDefinition[1]{\begin{quote}$\mathtt{#1}$\end{quote}}
\newcommand\TagDefinitionII[2]{\begin{quote}$\mathtt{#1}(\mathit{#2})$\end{quote}}

\begin{document}

\tableofcontents
\part{Architecture}

\chapter{Introduction}

\subsection{Background}
% * The gap - platforms lacking proper concurrency guarantees
%   vs. BEAM which is concurrency-first but pays perhaps too much
%   - poor in destructive updates / ephemeral data structures.
% * C++ and "safe" patterns (esp. regarding smart pointers);
%   "A virtual machine for C++"
% * Library/API documentation (, state of): names & types
%   but always not lifecycle and threading constraints.


\subsection{What is LinearVM?}
% An interface - in the form of a program representation
% and well-formedness criteria providing certain guarantees.

\section{Design philosophy}

\subsection{Priorities}

The design priorities are, in order from most important to less important:

\begin{enumerate}
\item Safety (type and concurrency)
\item Flexibility
\item Simplicity
\item Performance (cheapness in space and time).
\end{enumerate}

\subsubsection*{Safety}
Safety is paramount. The only (lamentable) exception is the
necessary support for using native libraries, i.e. having a Foreign
Function Interface (FFI). Ideally, development tools should provide
good support for stress testing FFI bindings.

\subsubsection*{Flexibility}
It should be easy to model a wide variety of languages and paradigms on the
virtual machine -- to the extent, of course, that the features to be modelled
are type and concurrency safe.
Furthermore, since it's still early days and there's little
experience with using linear types to implement real systems,
we have no desire to fix decisions prematurely.

\subsubsection*{Simplicity}
Quoth Dijkstra, ``Simplicity is prerequisite for reliability''.
Quoth Sessions paraphrasing Einstein: ``Everything should be as
simple as it can be but not simpler''.
Given the constraints set down by safety and flexibility, we strive
for simplicity. This is a fine balance.

One core source of simplicity is following the ``provide mechanism
rather than policy'' design rule: Including the general principles in
the VM, but leaving specific details to be added on top of it later.

\subsubsection*{Performance}
Speed and compactness are not primary goals. We believe that with the
right basis, and suitably strong code invariants, fast execution can
come later. (Even Javascript has been pushed far these days.)

If the above design priorities mean that program details get more explicit
(e.g. involve more steps),
then so be it -- at least as long as it's no worse than a constant factor.
(This overhead is akin to that of proof-carrying code [TODO: link]).


\subsection{Still early days/Core business}
As mentioned, there's little experience with linearly typed systems in
practice.

A theme which may become recurrent is that a design decision will be
resolved with ``we don't know enough about the real-world uses of this
yet -- so we'll effectively delay this decision by providing a suitably
general mechanism.''

Once some experience has been gained, and the real-world uses are
better understood, such design decisions may be revisited.


\section{Requirements}
The virtual machine should have a type system which
enforces type safety and safety against data races.
%Deadlock prevention may also be added, although that is less certain
%at this point.

The type system should be enforced by a bytecode verifier.

In order to be of practical use, the VM should have a FFI (foreign
function interface).  This is of course in itself a compromise of
type integrity and so forth, but it appears to be a practical necessity
-- all VMs
in practical use have it.  The linear type system would presumably
support FFI better than most other mechanisms, given that linear types are good at
describing management of resources such as entities provided by
foreign libraries.

\chapter{Design basics}
% All values are linear at the outset
% I.e. copying & disposal are explicit
% Consequences for exceptions

In the following, we sketch the most immediate consequences of the
requirements and priorities.

Design decisions tend to interact.
The following is a list of choices, approximately ordered by scope and
implications (wider and more basic decisions first).

\section{Linear types}
Firstly, supporting the combination of data race safety and imperative
paradigms leads to the use of linear types.

Modelling certain things in a concurrency-safe way becomes a great deal simpler
once you use linear types.
Choosing to support on linear types, and to rely on them for enforcing
certain invariants, is the basic design decision which makes LinearVM
stand apart.

Design decisions often interact with each other; this basic design choice
influences most other aspects of the VM design.

But should all values be regarded as linear, or only certain types? Numbers,
for instance, can be copied and destroyed without any special action.

We decide to treat all values as linear, for the following reasons:
\begin{itemize}
\item Non-linear types can easily be built on top of linear ones, by adding
  the trivial ``copy'' and ``dispose'' operations.
\item While number types which fit into the underlying machine's registers
  can be copied and destroyed without any special action, because they
  don't involve any external resources, the same is not true of
  e.g. ``bigints'' which involve some allocated memory.
  Thus, some number types are more linear-ish than others.
\item By treating all values as linear, we'll learn faster how to cope with
  different patterns of access and manipulation of linear values, and
  make them work for as wide a range of circumstances as the case allows.
\item Finally, there is simplicity in not having to deal with two
  kinds of values.
\end{itemize}

Thus, not only do we choose to support linear types -- we make them the
norm; in essence, all types are linear, and copying and destroying of values
is only possible if explicitly allowed.

\section{Layering}
The bytecode-verification algorithm is central to this project.
For security and robustness, it should be made as simple as the
typing rules allow -- yet flexible enough, of course, for real-world
programs.

What is least likely to change? The type system and typing rules, the
set of primitive types, or the set of operations on values?

The type system and its typing rules are presumably the most stable of
these three -- the most timeless part. Both the set of concrete types
and the set operations may well evolve over time.

Therefore, the type system and its verification algorithm is regarded
as the most fundamental component. Further up on the stack are the set
of types and further up yet the set of operations. These are less
fundamental to the degree that they're not actually fixed by the core
LinearVM design.

This serves two purposes which can be characterised as simplicity and flexibility:
Firstly, it simplifies the verifier, in that it does not need to incorporate
knowledge of all the primitve types and all the operations on them.
Secondly, it gives flexibility: the set of primitive types can be adjusted
more easily.
This design choice follows the separation design principle of ``mechanism,
not policy'' which reflects that it is beneficial to arrange software so
that stable, general parts are lower in the stack than parts which are
more specific and more prone to change.

Note that this architecture, in which the verification algorithm is
more or less ignorant about the concrete types and operations,
is in contrast to e.g. the JVM and CLR designs, in which the
byte-code verifier has hard-coded knowledge of a set of primitive types
and certain operations on them.

\section{Exceptions}

Linear types and exceptions don't play well together.
For standard exception propagation to work, one must assume that
all stack frames -- and therefore all values -- can be disposed of at will.
That is not the case, however, for linear values.

One might think up complex approximations to exception propagation.
In LinearVM, however, we take the simple solution:\footnote{
At least for the present. If experience shows that there are patterns to how exceptions are handled, then that might be supported more directly at a later point.}
At the VM level, \emph{exceptions do not exist} -- although they can be added on
a higher level of abstraction.
(This is quite like, at the level of physical processors, there is no
``exception'' concept.)

Operations which can either complete succesfully or terminate exceptionally
will be modelled using functions with \emph{multiple exits}.

\section{Multiple function exits}
Functions can be formalized as taking either a list of
parameters (like e.g. in the Algol language family) or always a single
parameter, which may be a tuple (like in the ML language family).
Equivalently, functions can be formalized as returning either through
a single exit (where the return type may be a sum type), or through
one of a list of exits.\footnote{
To be clear: this is orthogonal to whether each exit returns a single
value or a list of tuples.}

For some reason,\footnote{
  We expect that this hole in the design space to have syntactic roots
  -- that is, to be related to the syntax of function calls:
  the language-level lack of structured ways to elegantly specify
  multiple continuations (or rather, the separation of constructs with
  multiple continuations -- if, switch, try -- from the function call
  construct has rubbed off
  on the implementation-level designs of function call mechanisms.
} the latter option -- multiple exits -- is rarely if
ever seen except as hidden in the special case of exceptions, in which
case there are (implicitly) two exits: normal and exceptional termination.

LinearVM departs from normal designs on this point: while multiple
function exits would work well also with non-linear type systems,
the need for explicit representation of exception paths in particular
make it very attractive to have multiple function exits supported at the
lowest level.
(The alternative being to wrap and unwrap both exceptional and normal
return values as sum types whenever a function may generate an exception.)

\smallskip
Interestingly, with the inclusion of multiple function exits, certain
common machine instructions lose their special status: branching instructions
can be replaced with functions. This has implications for the
simplicity of the instruction set, as described in the chapter on architecture.

\section{Borrowing}
\label{sec:design-borrowing}
As mentioned, all values are linear.  Let's qualify this: Every value
has a primary owner, but there can be secondary references to a value,
subject to certain restrictions.

There are several advantages to linear values, including:
\begin{itemize}
\item no risk of race conditions
\item simple resource management: you know when the value is no longer live
\item no tracing garbage collector is required
\end{itemize}

\noindent
However, a \emph{purely} linear type system has it drawbacks as well:
\begin{itemize}
\item Subvalues of a composite data structure can't be accessed
  without deconstructing the value. This is particularly bothersome
  for deep field accesses.

\item Values can never be shared, even for read-only purposes; instead,
  values must be copied deeply.

\item Certain data structures, such as doubly linked lists, or even
  singly linked lists with a pointer to the tail, or other data
  structures which have several pointers to one value, can not be
  expressed -- at least not with the same (asymptotic) performance,
  using purely linear types.
\end{itemize}

To overcome these deficiencies while keeping the advantages of linear types,
LinearVM has the concept of \emph{borrowing}.

A borrowed value is a secondary reference to a value.
While there is always just a single primary reference to a given value
-- from its linear owner -- there can be several secondary references.

Borrowing comes in two flavours: \emph{exclusive} and \emph{shared} borrowing.

An owning reference, an exclusive borrowing, and a shared borrowing differ
with respect to what can be done with them:
\begin{itemize}
\item A \emph{shared} borrowing can be used for \emph{reading}: to test,
  probe, and copy the value or parts thereof.
\item An \emph{exclusive} borrowing can furthermore be used to \emph{modify} the value,
  as long as the value's type remains unchanged.

\item An \emph{owning} reference can furthermore be used with
  operations which \emph{consume} the value, including operations
  which transform it into a value of a different type or which dispose
  of the value altogether.
\end{itemize}

The mechanisms for obtaining and manipulating borrowings impose restrictions
so that the central invariant is maintained that:
\begin{quote}
  As long as a value can be accessed through one reference,
  it can not be modified through another reference.
\end{quote}

\subsection{Local borrowing} %% Naming ??
\label{sec:design-local-borrowing}
A common special case of borrowing is when a function borrows a value
from its caller -- or, in term of owners, when a function borrows a value from
the stack frame below it -- and returns it on exit (on every exit path).

In this case, there is a shortcut which uses the same underlying mechanism,
but which reduces book-keeping.
It is called \emph{local borrowing}, and
is denoted in the function's signature with a ``\texttt{\&}'' or
``\texttt{\&!}'' prefix to the type (for shared resp. exclusive
borrowing), like this:
\FuncSignature{add}{(a:\&int, b:\&int) \to int}

The advantage of local borrowing is that both the ``borrow'' and the
``return'' operation are made part of the actual function call.
The borrowed value is returned implicitly. The use of local borrowing requires,
however, that at all return points the borrowed value is present in the
same register as it was in when the function was entered.
(Note, though, that this does not mean that the borrowed value must be
unchanged since the function was entered.)

Compare with the function signature of a version of \emph{add} which does
not use local borrowing:

\FuncSignatureII{add}
{(a:borrow\_shared(ID1,int), b:borrow\_shared(ID2,int))}
{(a:borrow\_shared(ID1,int), b:borrow\_shared(ID2,int), result:int)}

Local borrowing can obviously only be used when borrowing and returning
follows the call stack. In other cases, and when the borrowings are
put in data structures off the stack, regular ``full'' borrowing must be used.
In both cases, though, the same underlying mechanism is used, as far
as the type system is concerned.

\chapter{Virtual machine architecture}

\section{Registers}
The LinearVM virtual machine has a register-based architecture.
There is a conceptually unlimited number of registers, and each
register can hold a value of any type.

The virtual machine uses register frames, which means that each
function gets a fresh set of registers (they correspond to local
variables) and does not have to save registers for its caller or callees.

\section{Instruction set}
To a certain extent, the interesting thing is not so much the
concrete operations which the VM provides, as the type safety that it
provides.
Therefore, the design weighs the importantance of general mechanisms
over specific data types and operations.

\smallskip
One concrete way in which this is done, is to take following quote:
\begin{quote}
Static methods are a great way to add opcodes.\\
\raggedleft
-- James Gosling
\end{quote}
to an extreme: only leave opcodes which can't be made into functions.
This is advantageous in terms of byte-code verifier simplicity, since
the verifier does not need to incorporate information about the
existence and signatures of these operations.
The focus of the instruction set is to describe data and control flow.

Operations which can be made into function are relegated to the
standard library. This includes all arithmetic and conversion
operations, of course, but far more than that.

Even operations with multiple outputs, such as product-type deconstruction,
can be made into functions, if functions are allowed to have multiple
return values.

And even operations with multiple continuations/outcomes, such as
sum-type deconstruction, can be made into functions, if functions are
allowed to have multiple exits -- which they are.
(Two-way branches based on boolean values are a special case of this,
since the boolean type is a simple sum type.)

\smallskip
What remains, then, is:
\begin{itemize}\def\itemsep{0pt}
\item A function call instruction -- more powerful, perhaps, than is
  usually seen, as it must handle both multiple inputs, multiple
  exits, and multiple return values for each exit.
\item A return instruction -- which can return multiple values to one
  of possibly several exits.
\item A constant loading instruction.
\item An instruction labelling mechanism.
\end{itemize}



% Minimal instruction set
% - MFEs make branching instructions non-basic/functionalizable.
% Registers: contain any type (of known type)
% TODO


\part{Type System}

% Type constructors
% Type properties
% Typing rules
% Verifier
% Motivational typing examples:
% - bignums
% - concurrent queues
% - ref-counted objects
% - locks
% (- sparse matrix...)

\part{Types and Operations}

\chapter{Introduction}
\chapter{Small functions}
% Pseudo-instructions not tied to any specific type or type constructor.

\texttt{goto : () $\to$ (never) | ()}

\chapter{Small constructed types}

\section{Common type tags}

\subsection*{Identity}
It is at times useful to tie a type to a specific value of another type.
The \emph{identity tag} \texttt{virtualvm.basis.id} is used for this.

In \texttt{linearvm.basis}:\\
\texttt{give\_identity : (T) $\to$ $\exists$ X: \tyBless(virtualvm.basis.id, X, T)}

\subsection*{Number relations}
Especially in connection with array subscript checking it is useful to encode
relations among integral numbers.
The following type tags exist for that purpose:

\begin{itemize}
\item \texttt{linearvm.basis.equal}
\item \texttt{linearvm.basis.inequal}
\item \texttt{linearvm.basis.less\_than}
\item \texttt{linearvm.basis.less\_or\_equal}
\end{itemize}

The use of these tags will be described in more detail later in this chapter.

\section{Void}
``Void'' (\texttt{linearvm.basis.void}) is the unit type.
It is defined as the record type with no fields:
\begin{center}
  \verb|record {}|
\end{center}
The official name is just an alias.

Void is special in two respects:
\begin{enumerate}
\item it can be represented in zero bits;
  it is therefore not necessary to allocate space for void values at
  runtime, which means that they are in a sense ``free''.
  For this reason, void and derivatives (by $\tyBless$ and
  $\tyCurse$) are at times used to represent proofs and obligations.
\item it is trivial to construct and dispose of;
  it is therefore used (in conjunction with $\tyPad$) to represent e.g.
  uninitialized or emptied fields.
\end{enumerate}

Note that void is not usually used as a return type: there is a
difference between returning void (a record value with zero fields)
and returning zero values.

\section{Never}
``Never'' (\texttt{linearvm.basis.never}) is the bottom type.
It is defined as the variant type with no flavours:
\begin{center}
  \verb|variant {}|
\end{center}
The official name is just an alias.

Never is special in that a value of this type can not be constructed.
Function calls which have never as their first return value for some return
path are treated specially by the verifier: it assumes that the return
path in question will never be taken, and disregards that return edge
in the control-flow graph.
This is used in connection with the ``goto'' (\texttt{linearvm.basis.goto()})
pseudo-instruction.

\section{Boolean}
``Boolean'' (\texttt{linearvm.basis.boolean}) is the type of truth values.
It is defined as
\begin{center}
  \verb|variant {false: void, true: void}|
\end{center}

\subsection*{Operations}

\begin{itemize}
\item \texttt{and : (boolean,boolean) $\to$ boolean }
\item \texttt{or : (boolean,boolean) $\to$ boolean }
\item \texttt{not : (boolean) $\to$ boolean }
\item \texttt{if : (boolean) $\to$ void $|$ void }
  \\ This is equivalent to an ordinary ``case''
  (\texttt{linearvm.variant.case}) on the variant type.
\end{itemize}

\section{Proof types -- Relations among integers}

The following helper types are defined to encode knowledge of the relation of
integer values:

\medskip
\noindent
$\mathtt{linearvm.basis.equal}[X_1,X_2] := \tyBless(\mathtt{linearvm.basis.equal}, X_1,X_2, \mathtt{void})$
$\mathtt{linearvm.basis.inequal}[X_1,X_2] := \tyBless(\mathtt{linearvm.basis.inequal}, X_1,X_2, \mathtt{void})$
$\mathtt{linearvm.basis.less\_than}[X_1,X_2] := \tyBless(\mathtt{linearvm.basis.less\_than}, X_1,X_2, \mathtt{void})$
$\mathtt{linearvm.basis.less\_or\_equal}[X_1,X_2] := \tyBless(\mathtt{linearvm.basis.less\_or\_equal}, X_1,X_2, \mathtt{void})$

\subsection*{Operations}

In \texttt{linearvm.basis}:\\
\begin{tabular}{r@{}l}
\texttt{compare : \&id(X1,int), \&id(X2,int) $\to$}
& \texttt{less\_than[X1,X2]}\\
$|\;$ & \texttt{equal[X1,X2]}\\
$|\;$ & \texttt{less\_than[X2,X1]}\\
\end{tabular}
\\
\begin{tabular}{l@{}l}
\texttt{commute : \&equal[X1,X2]} & \texttt{$\to$ equal[X2,X1]}\\
\texttt{relax : \&less\_than[X1,X2]} & \texttt{$\to$ less\_or\_equal[X1,X2]}\\
\texttt{relax : \&equal[X1,X2]} & \texttt{$\to$ less\_or\_equal[X1,X2]}\\
\end{tabular}
\\
\begin{tabular}{l@{}l}
\texttt{combine : \&less\_or\_equal[X1,X2], \&less\_or\_equal[X2,X1]}
& \texttt{$\to$ equal[X1,X2]}\\
\end{tabular}

\chapter{Integer types}
\chapter{Floating-point types}
\chapter{Bytestrings}


% Allocate: ?
% get_size: bytestring -> uint



\chapter{Strings}
\chapter{Borrowing}
As described in \fref{sec:design-borrowing}, borrowing is used when
more than one reference to a value is needed at the same time, and also
when a value passed to a function is returned from it through all exit paths.
As mentioned, there are two types of borrowing: shared (read-only) and
exclusive (read-write).

Beside the operations listed here, the parameter passing mechanism of
\emph{local borrowing} (\ref{sec:design-local-borrowing})
converts linear values to and from borrowed ones.
% TODO: That mechanism should be described in detail somewhere.

\subsection*{Operations}
Borrowing exclusively:

\FuncSignature{borrow}{T \to \exists ID: (curse(borrowed, ID, T), borrow\_excl(ID, T))}
\FuncSignature{unborrow}{\forall ID: (curse(borrowed, ID, T), borrow\_excl(ID, T)) \to T}

\noindent
Borrowing shared:

\FuncSignature{borrow\_shared}{T \to \exists ID: (curse(borrowed, ID, T), borrow\_shared(ID, T))}
\FuncSignature{unborrow\_shared}{\forall ID: (curse(borrowed, ID, T), borrow\_shared(ID, T)) \to T}

\noindent
From exclusive to shared:

\FuncSignature{weaken}{borrow\_excl(ID, T) \to (borrow\_shared(ID, T), strengthen\_token[ID,T]}
\FuncSignature{strengthen}{(borrow\_shared(ID, T), strengthen\_token[ID,T]) \to (borrow\_excl(ID, T)}

\noindent
From shared to twice shared:

\FuncSignatureII{split}
{borrow\_shared(ID, T)}
{(borrow\_shared(ID1, T), borrow\_shared(ID2, T), join\_token(ID1,ID2,ID))}

\FuncSignatureII{join}
{(borrow\_shared(ID1, T), borrow\_shared(ID2, T), join\_token(ID1,ID2,ID))}
{borrow\_shared(ID, T)}


% \noindent
% Temporarily changing type of an exclusive borrowed value:
% For all $T$ and $T'$ with the same type representation:
% \FuncSignatureII{retype}{borrow\_excl(ID, T)}{\exists ID': (borrow\_excl(ID', T'), retype\_token(ID',T',ID,T))}
% \FuncSignatureII{de\_retype}{(borrow\_excl(ID', T'), retype\_token(ID',T',ID,T))}{borrow\_excl(ID, T)}
% TODO: Not stable API. Actual type change happens at e.g. record field store. Address this.


% borrow_ro : T -> exists(ID): (curse(borrowed_ro, <ID>, T), borrowed(<ID>, curse(read_only, T)))
% breturn : all(ID): (curse(borrowee, <ID>, T), borrowed(<ID>, T)) -> T

\chapter{Arrays}

Arrays are integer-indexed sets of values.
Array indexing takes constant time.
Arrays are created with a fixed, finite capacity.

\section{Entry validity}

What makes arrays difficult to model is the absence of general null values:
Each valid array entry must contain *something*, so when we create an
array of some non-fixed size, then if all the entries are valid, we
have just created a non-fixed number of values.
This won't do in the general case, with any type.

We therefore associate with each array value not just a length, but
also a measure of how much of the array is initialized, i.e. which of
the entries are valid.

We restrict this set to be a \emph{prefix} of the array (at least at
present -- support for initialized suffixes, or circular buffers, can
be added later).

\section{Bounds checking}
In any safe language, array indexing must be subject to \emph{bounds checks}
-- verification that the integer used for indexing does in fact correspond to
an entry within the finite range of the array.

It is often the goal of implementations of such languages to optimize
away these checks, as they cost run-time performance.

Because of the flexiblity of LinearVM's type system, we have the option
of requiring index validity proof on access to arrays.

% Allocation
% Deallocation
% "Valid" pointer adjustment (first store & final fetch)
% Update - at valid index
% Fetch - swap
% Fetch - borrow
% Swap two elements

\chapter{Records}
% create, fetch, store, borrow, deconstruct

\chapter{Variants}
\chapter{VTables}
% call_<F> : (vtable(\{\ldots, f: (a_1, \ldots a_n) -> o_1 \| \ldots \| o_m),
%            a_1, \ldots a_n) -> o_1 \| \ldots \| o_m.
% dispose: vtable(\ldots) -> ()
\chapter{Existentials}
% create : (T) -> (\tyExist(v,T)) when ...?
% unpack!

\chapter{Pointers}
% store: (T, ptr(pad(void, T))) -> ptr(T)
% fetch: ptr(T) -> (T, ptr(pad(void, T))) %% TODO: How to make work for different kinds of pointer?
% - ptr(V, T) ?

\chapter{Memory allocation}

% malloc: () -> bless(mallocated, ptr(pad(void, T)))
% free: bless(mallocated, ptr(pad(void, T))) -> ()

\chapter{Auxiliary pointers}
% Tentative... & WIP
``Auxiliary pointers'' is a mechanism for referring to locations within
a data structure, and for doubly-linked data structures.
Auxiliary pointers are a certain kind of non-owning pointers.
% TODO: Relation to borrowed pointers: double-ended-ness.

\begin{itemize}
\item Auxiliary pointers can be set or unset. They start out being unset.
  An auxiliary pointer can only point to one value at a time.
\item Auxiliary pointers are parametrized with a type $T$.
  When set, an auxiliary pointer holds a pointer to $T$.
\item If an auxiliary pointer points to a value, that value must exist,
  and must keep existing while being pointed at.
  A consequence of this is that, in a certain sense, the pointee must ``know''
  that is may be a target of the auxiliary pointer.
\end{itemize}

\noindent
There are four entities involved when manipulating an auxiliary pointer:
% TODO: There should in fact be four: the pointee should have a state token too. This improves and simplifies the interface (making bind2/unbind2/decurse unnecessary).
\begin{enumerate}
\item The auxiliary pointer itself.
\item The value which the auxiliary pointer is to point to (the pointee) --
  or, more precisely, the owning pointer of that value.
\item A token value which testifies either that the auxiliary pointer is unset,
  or that is pointing at the value.
\item A token value which testifies whether the pointee is being
  pointed to or not.
\end{enumerate}

\subsection*{Types}
The package containing the functions for manipulating auxiliary pointers is
\texttt{linearvm.auxpointer}.

\smallskip
The types in the package are parametrized by four kinds of type variables, named:
\begin{quote}
\begin{description}
\item[T] -- Value type
\item[P] -- Pointer flavour
\item[APID] -- Auxiliary pointer identity
\item[VID] -- Value identity
\end{description}
\end{quote}


\smallskip
\noindent
The package defines the types:

\TypeDefinition{auxpointer[APID,T]}
\TypeDefinition{auxpointee[VID,P,T] = pointer(curse(\texttt{auxpointee}(VID),P),T)}

\noindent
The token types:

\TypeDefinition{unbound[APID]}
\TypeDefinition{bound[APID,VID]}
\TypeDefinition{unbound\_pointee[VID]}
\TypeDefinition{bound\_pointee[APID,VID]}

\noindent
And the tag:
\TagDefinitionII{auxpointee}{VID}


\section*{Operations}

\subsection*{Creation and disposal}
The first step in using an auxiliary pointer is creating it.
At the same time, a token value is created which testifies that the
auxiliary pointer is unset:
\FuncSignature{create}{() \to \exists APID.(auxpointer[APID,T], unbound[APID])}

\noindent
Pointers are marked explicitly as potential pointees:
\FuncSignatureII{mark\_pointee}
{pointer(P,T)}
{\exists VID.(auxpointee[VID,P,T], unbound\_pointee[VID])
}

\noindent
Conversely, an unbound auxiliary pointer can be disposed of:
\FuncSignature{dispose}{(auxpointer[APID,T], unbound[APID]) \to ()}

\noindent
And a potential pointee which is not currently bound can be unmarked:
\FuncSignatureII{unmark\_pointee}
{(auxpointee[VID,P,T], unbound\_pointee[VID])
}
{pointer(P,T)}

\subsection*{Binding and unbinding}
An unbound auxiliary pointer can be bound to an unbound pointee.
This marks both the pointer and the pointee as bound:

\FuncSignatureII{bind}
{(\&auxpointer[APID,T], unbound[APID], \&auxpointee[VID,P,T], unbound\_pointee[VID])}
{bound[APID,VID], bound\_pointee[APID,VID]}

\noindent
Unbinding occurs similarly:
\FuncSignatureII{unbind}
{(\&auxpointer[APID,T], bound[APID,VID], \&auxpointee[VID,P,T], bound\_pointee[APID,VID])}
{(unbound[APID], unbound\_pointee[VID])}

\subsection*{Accessing auxiliary pointers}

A bound auxiliary pointer can be dereferenced:

\FuncSignature{deref}{
  (\&auxpointer[APID,T], \&bound[APID,VID])
  \to
  curse(\mathtt{borrow}, auxpointee[VID,P,T])}

(TODO: get borrowing right: lock auxpointer for the duration.)

% TODO: ensure that you can't bind-access-unbind-deallocate-deaccess...

% create: () -> \exists X:(revptr(X, T), permission_to_set(X))
% set: \forall X,PT: (permission_to_set(X), bless(ID,ptr(PT,T))) -> rev_ptr_set(ID), ???bless(ID,ptr(PT,T))

\subsection*{Paradox proofs}

At times, it is useful to be able to prove that certain invariants are
guaranteed to hold because of a combination of available resources.
The following functions are provided for this purpose:

\FuncSignature{paradox\_bound\_pointer}{
  (bound[APID,VID], unbound\_pointee[VID]) \to \mathtt{never}}
\FuncSignature{paradox\_bound\_pointee}{
  (unbound[APID], bound\_pointee[APID,VID]) \to \mathtt{never}}

\chapter{Paradoxes}
The package \texttt{linearvm.paradox} contains types and operations
with the primary purpose of expressing invariants in paradox form --
i.e. deriving a value of the type \texttt{never}.

\section{Tick-tock}
Tick-tock is a two-phase token type.
It is defined in the package \texttt{linearvm.paradox.ticktock}.
The fundamental invariant is that the value associated with a given identity
is at any point either tick or tock, but never both.
Tick-tock can be used e.g. for expressing temporary suspension of an invarant.

\subsection*{Types}
\TypeDefinition{tick[ID] = tag(\texttt{tick}(ID), void)}
\TypeDefinition{tock[ID] = tag(\texttt{tock}(ID), void)}

\subsection*{Operations}
\subsubsection*{Creation and disposal}
\FuncSignature{create\_tick}{() \to \exists ID.tick[ID]}
\FuncSignature{dispose}{ID.tick[ID] \to ()}

\subsubsection*{Phase shift}
\FuncSignature{to\_tock}{(ID.tick[ID]) \to ID.tock[ID]}
\FuncSignature{to\_tick}{(ID.tock[ID]) \to ID.tick[ID]}

\subsubsection*{Paradox proofs}
Given the invariant that the value associated with a given identity
can never at the same time be tick and tock, we get the paradox proof:
\FuncSignature{paradox}{(\&ID.tick[ID], \&ID.tock[ID]) \to \texttt{never}}

\chapter{Miscellaneous}

\section{Panic}

There is a (restricted/restrictable) equivalent to C's \texttt{abort()} function.
It is defined in the package \texttt{linearvm.panic}.

\subsection*{Types}
\TypeDefinition{permission\_to\_panic = bless(\texttt{permission\_to\_panic}, \texttt{void})}

\subsection*{Operations}
\FuncSignature{panic}{permission\_to\_panic \to \texttt{never}}


\part{File Format}

\part{Implementation}

\section{Types with special treatment}
% TODO: where to place this? Never-like types is a verifier thing; void-like types is mostly an implementation thing

Certain types are handled specially by the verifier or at runtime.

\subsection{Void and void-like}
\emph{Void-like} types are certain types whose values can be
represented in zero bits.
They are defined as follows:

{
%\def\voidlike #1{{\ensuremath\textsf{void-like}(#1)}}
\def\voidlike #1{\ensuremath{#1\ \textsf{void-like}}}
\begin{tabular}{cc}
\\
$\voidlike{\tyRecord\{\}}$
& ${\voidlike{T}} \over {\voidlike{\tyBless(\_, \_, T)}}$
\\
\\
${\forall i : \voidlike{T_i}} \over {\voidlike{\tyRecord\{f_1 : T_1, \ldots, f_n : T_n\}}}$
& ${\voidlike{T}} \over {\voidlike{\tyCurse(\_, \_, T)}}$
\\
\\
${\voidlike{T}} \over {\voidlike{\tyBorrow(\_,T)}}$
\\
\\
\end{tabular}

Since these types only contain one value, it is not necessary to
allocate space for void values at runtime, and void-like values are
not present at runtime.
The value's part has been played out at verification time, where it
contributed with its type.

\subsection{Never and never-like}

\emph{Never-like} types are certain types which contain no values at all.
They are defined as follows:

{
%\def\neverlike #1{{\ensuremath\textsf{never-like}(#1)}}
\def\neverlike #1{\ensuremath{#1\ \textsf{never-like}}}
\begin{tabular}{cc}
\\
$\neverlike{\tyVariant\{\}}$
& ${\neverlike{T}} \over {\neverlike{\tyBorrow(\_,T)}}$
\\
\\
\end{tabular}

Since values of these types do not exist, ownership of such a value along
a given control-flow path is evidence that that particular control-flow path
can never be reached.

Therefore, any function which returns a never-like value in one of its return
paths can be assumed to never return along that particular path.
The verifier treats specially any function call which has a value of a never-like type as its \emph{first} return value for some return path:
because the path can never be taken, the verifier disregards that
return edge in the control-flow graph.

(The restriction to only the first return value is for simplicity and
speed of the verifier, and does not lose generality: if any but the
first return value is never-like, then the program can call the
identity function on that value as its next instruction and trigger
the special handling.)

\medskip
One practical example of this effect is the \texttt{goto} pseudo-instruction,
which is a function with the signature
\begin{center}
  \texttt{goto : () $\to$ (never) | ()}
\end{center}
which means that it will never return through its normal return path
to the following instruction.

\end{document}
