\documentclass[a4paper]{report}
\usepackage{amsmath}

\newcommand\arity{{\ensuremath\textit{arity}}}
\newcommand\fixedsize{{\ensuremath\textit{fixed\_size}}}

% Type constructors:
\newcommand\tyImport{{\ensuremath\textit{import}}}
\newcommand\tyVar{{\ensuremath\textit{var}}}
\newcommand\tyApp{{\ensuremath\textit{app}}}
\newcommand\tyBless{{\ensuremath\textit{bless}}}
\newcommand\tyCurse{{\ensuremath\textit{curse}}}
\newcommand\tyPtr{{\ensuremath\textit{ptr}}}

\begin{document}

\chapter{Verification algorithm}
\label{chap:verifier}

Integrity of a LinearVM instance is dependent on the well-formedness
of the code it loads.
An important part of well-formedness is type consistency.
At load time, the code files to be loaded are subject to \emph{code
  verification}, which is a series of checks that certain properties are fulfilled.

\section{Trade-offs}
As it is known from the field of Proof-Carrying Code, there is a
three-way trade-off between
\begin{description}
\item[Expressiveness] -- what properties can be proved and verified;
\item[Verifier computing power] -- how much of the proofs the verifier is to reconstruct; and
\item[Explicitness] -- how much of the proofs are contained in the
  program representation.
\end{description}

With stronger properties come usually larger proofs, which again
requires either the verifier to be smarter or the program
representation to be larger.

With the expressiveness dictated by other requirements, we thus have a
time/space trade-off between less compact programs and less simple and
fast verification algorithms.

LinearVM's design tends to favour simplicity over compactness.
In particular, it aims to avoid:

\begin{itemize}
\item Fix-point algorithms.
  %% JVM changed partly to a similar strategy, following "lightweight
  %% bytecode verification" (http://evarose.net/thesis-submitted.pdf),
  %% also known as "split verifier".
\item Proof search.
  Where more than one subtyping rule or similar is applicable, the
  code must carry sufficient information to determine which, in bounded time.
\item Type unification.
  Rather than unification, verification criteria are defined in terms of
  \emph{assignability}. This means that the verifier will not have to
  reconstruct types at control flow joins etc.

\end{itemize}


\section{Checks}

\begin{description}
\item[Syntactic conformance]
  The file must conform to the format described [in the ``File Format'' chapter].
\item[Index validity]
  All indices into sections must be valid.
  This includes IDs of:
  \begin{itemize}
  \item Simple symbols
  \item Qualified symbols
  \item Types
  \end{itemize}

  It also includes byte-offset pointers into:
  \begin{itemize}
  \item Symbol name source
  \end{itemize}
\item[Back-pointing index validity]
  Certain indices pointing from one section into that same code
  section are constrained such that the refered-to item must occur
  prior to (i.e., have a lower index than) the refering item.

  This includes:
  \begin{itemize}
  \item The ``parent'' element of a qualified-symbol (if non-null).
  \item Any type refered to in a type definition.
  \end{itemize}
\item[Liveness consistency of functions]
  Each function must be liveness consistent, as described [below].
\item[Type consistency of functions]
  Each function must be type consistent, as described [below].
\end{description}

\section{Liveness consistency}
At any point during the execution of a function, any register of the
function is in one of three states:
\begin{itemize}
\item Dead, i.e. not containing a value;
\item Live w/ reference, i.e. containing a reference;
\item Live w/ value, i.e. containing a value.
\end{itemize}

\emph{Liveness consistency} is a property of a function, stating that
regardless of the control flow path taken through the function,
\begin{itemize}
\item No register is read from if it is dead;
\item No register is written to if it is live w/ value;
\item When the function returns, no register is live w/ value.
\end{itemize}

%%% Lattice:
%%%        Top
%%%       /   \
%%%    Dead    \
%%%      |      \
%%%  Live-ref  Live-value
%%%       \      /
%%%        Bottom

\subsection{Lattice}
The join function for liveness state is defined as follows:
\begin{equation*}
join(L_1, L_2) =
\begin{cases}
  L_1 & \text{if } L_1 = L_2,\\
  dead & \text{if } \{L_1,L_2\} = \{dead, live-ref\},\\
  top & \text{otherwise}.
\end{cases}
\end{equation*}

\subsection{Boundary conditions}
At the beginning of the function, any register corresponding to a
value-parameter is live w/ value; any register corresponding to a
reference-parameter is live w/ reference; and any other register is
dead.

\subsection{Forward flow analysis}
The instructions are handled as follows:
\begin{description}
\item[Label] 
\end{description}



\section{Type consistency}

\subsection{Register-set types}
%Type

\end{document}
