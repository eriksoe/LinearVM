\documentclass[a4paper]{book}

\begin{document}
\chapter{Examples}

The examples in this chapter is mainly inspired by the Erlang
programming model.

\section{Message inbox}

Message queue with Erlang-style selective receive: First message to
pass a given criterion (if any) is taken out of the queue.

% TODO: How do we even represent a queue with a tail pointer?
% Can borrowing be made flexible enough for that?
% Operations:
% - create empty.
% - insert at tail.
% - traverse from head, using links, pick first satisfying criterion out.
%
% +------+
% | head ----->A--->B---->C
% |      |                ^
% | tail -----------------'
% +------+
% Can this be solved with auto-back-pointing?
%
% => "Auxiliary pointers"

$$
mqueue[T] := record(TODO)
$$

\end{document}
