\documentclass[a4paper]{report}
\usepackage[latin1]{inputenc}
\usepackage[T1]{fontenc}

\begin{document}

\tableofcontents

\chapter{Introduction}
This specification defines the ``LinearVM'' abstract machine.

\section{Goals and non-goals}
The design of LinearVM aims to\ldots
\begin{itemize}
\item be usable for general-purpose programming

\item support the creation of abstraction which do not leak
  wrt. concurrency and memory management issues

\item support efficient implementations

\item be flexible, yet small
\end{itemize}

\noindent
The design explicitly does not aim to\ldots
\begin{itemize}
\item be competitive with existing VMs in terms of raw performance
\item run any programs written in existing languages
\item support arbitrary coding or language styles
\end{itemize}

In short, the design of LinearVM aims to be practical enough for
real-life uses, without compromising concurrency sanity. %? ...

% Scope: After compilation (no language or syntax). No implementation details, except in the form of suggestions.

\section{Conceptual overview}
Conceptually, a LinearVM system consists of
\begin{enumerate}
\item A file format
\item A validator
\item An executor
\item A set of native functions.
\end{enumerate}

A LinearVM program consists of one or more modules.
Each module exists in the form of the file format specified in this
text.

A virtual machine which is to run the program first loads and
validates each module; modules loaded in this manner is then made
available for the executor to run.  If a module file fails validation,
the virtual machine will refuse to load it.

\subsection{Code entities}

\paragraph{Names.}
Some code entities are named.  Entity names are arbitrary unicode
strings (refered to as \emph{symbols}).

\paragraph{Namespaces.}
Each named code entity belongs to a \emph{namespace}.  Namespaces are
hierachical; they are named by a sequence of symbols separated by
periods.

(The namespaces \texttt{linearvm} and \texttt{linearvm.basis} are reserved for LinearVM-defined entities.)

\paragraph{Functions.}
A central kind of code entity is the \emph{function}.
A function has a \emph{type signature} which describes the nature of its inputs and outputs, and a \emph{body} which consists of a piece of executable code in the form of a sequence of instructions.

\paragraph{Types.}
The type system is central to the way LinearVM works; in particular, it is central to code validation.
Types may be named.

\section{Imports and exports}
A named code entity may be either \emph{local} to the module, or \emph{exported} for use by other modules.
Code entitites which may be exported are: types and functions.

\section{Flexibility and non-commitment}
The design of LinearVM follows the design philosophy of ``mechanism, not policy''. Many things are left open by this spec, such as the set of available numeric types. Even basic operations such as addition fall outside the scope of the core LinearVM spec. Instead of specifying such types and functions, LinearVM provides mechanisms for defining them.

% Defers decisions

\section{Outline}
The rest of this text is structured as follows:

In chapter \ref{ch:execution}, the architecture and execution model of
the abstract machine is described.

Chapter \ref{ch:types} describes the type system.

Chapter \ref{ch:fileformat} describes the LinearVM file format.



\chapter{Program Executor}
\label{ch:execution}

The program executor of a LinearVM system is the core of the abstract machine.
It is register-based and function-based, and has a quite small instruction set.


\section{Architecture}
\begin{figure}[h]
  \centering
  %\begin{minipage}{10em}
  \begin{tabular}{rl}
    Instruction pointer &
  \begin{tabular}{|c|}
    \hline
    $IP$\\
    \hline
  \end{tabular}\\[0.5em]

    Registers &
  \begin{tabular}{|c|c|c|c|}
    \hline
    $r_1$ & $r_2$ & $\cdots\cdots$ & $r_n$ \\
    \hline
  \end{tabular}\\[0.5em]

    Return addresses &
  \begin{tabular}{|c|c|c|c|}
    \hline
    $c_1$ & $c_2$ & $\cdots$ & $c_m$ \\
    \hline
  \end{tabular}
  \end{tabular}
  %\end{minipage}

  \caption{Architecture of LinearVM machine}
\end{figure}

\section{Instruction set}

The machine's instruction set consists of the following instructions:

\begin{description}
\item[Load] -- load a constant into a register.
\item[Return] -- return from the current function.
\item[Call] -- call a function.
\item[Label] -- marks a point in the instruction stream.
\end{description}

The individual instructions are described in more detail in the following.

It is assumed that the instructions occur within a function of type
$$
(in_1, \ldots, in_n) -> (out_1, \ldots, out_{m_1}) \;|\; (out_1, \ldots, out_{m_2}) \;|\; \ldots \;|\; (out_1, \ldots, out_{m_k})
$$

\subsection{The Load instruction}
%% Short description:
The instruction
$$
\texttt{load } c, r
$$
stores into the register $r$ the constant $c$.
%% TODO: Long description.

\subsection{The Return instruction}
%% Short description:
The instruction
$$
\texttt{return } rp\; (r_1, \ldots, r_n)
$$
returns via the return path $rp$, with return values $r_1, \ldots, r_n$.

\paragraph{Preconditions:}
The arity of the return path $rp$ is $n$.
For $i \in 1 \ldots n$, the type of $r_i$ is assignable to the $i$th return
value type of $rp$.
For $i \in 1 \ldots n$, the value in $r_i$ is spendable.
value type of $rp$.
The set of available and spendable registers are precisely $\{r_i | i \in 1 \ldots n\}$.

\subsection{The Call instruction}
%% Short description:
The instruction
$$
\texttt{call } f\; (r_{io_1}, \ldots, r_{io_p});(r_{in_1}, \ldots, r_{in_n})
\rightarrow(r_{{out0}_1}, \ldots, r_{{out0}_{m_0}})
rp_1 (r_{{out1}_1}, \ldots, r_{{out1}_{m_1}})
\ldots
rp_k (r_{{outk}_1}, \ldots, r_{{outk}_{m_k}})
$$
calls the function $f$ with arguments from registers $r_{inout_1}, \ldots, r_{inout_p}$ and $r_{in_1}, \ldots, r_{in_n}$.
A number $k$ of alternative return paths are available to the called function;
$rp_i$ is the corresponding return label and
$r_{{outi}_1}, \ldots, r_{{outi}_{m_i}}$ the registers in which the
result values are placed before execution resumes at the return label.
$rp_0$ is implicit: it is the instruction following the call instruction.

\paragraph{Preconditions:}
Let $A$ be the set of occupied registers, and $S$ the set of
spendable registers.\\
Let $IO = \{r_{io_i} | i \in 1 \ldots p\}$.\\
Let $I = \{r_{in_i} | i \in 1 \ldots n\}$.\\
Let $O_j = \{r_{out j_i} | i \in 1 \ldots m_j\}$.\\

Then,
\begin{itemize}
\item all the inputs are distinct registers: $IO \cap I = \emptyset$, $|IO|=p$, $|I|=n$;
\item all the outputs are distinct registers: $\forall j: |O_j| = m_j$;
\item all the inputs are occupied: $(IO \cup I) \subseteq A$;
\item all the spent inputs are spendable: $I \subseteq S$;
\item all the output registers are empty at return time: $\forall j: O_j \cap (A \setminus I) = \emptyset$.
\end{itemize}

$S'_j := (S \setminus I) \cup O_j$.

%% TODO: Long description.

\subsection{The Label instruction}
%% Short description:
The instruction
$$
\texttt{label }\textit{name}
$$
marks the code position as label $\textit{name}$.



\chapter{Type System}
\label{ch:types}

\section{Terminology}

\begin{description}
\item[Ground type] A type which has no free type variables.
\item[Arity of a type] ($\mathit{arity}(T)$) The highest level of its free variables.
  Equal to the number of times the type would have to be
  subject to type application for it to become a ground type.
\item[Known-size type] A type which has a known run-time representation size.
\end{description}

\section{Type kinds}

\subsection{Imported types}
Types imported from other modules. These are opaque types from the
importing module's point of view.
$$ \mathit{import(name, arity)} $$

$\mathit{arity}(T) = \mathit{arity}$\\
$\mathit{known\_size}(T) = (arity=0) \lor ??$\\

%% TODO: It would be nice to be able to export structs etc., non- or
%% semi-opaquely - but that requires a bit of thought to get right.

\subsection{Product types (records)}
$$ \mathit{product((symbol_1, type_1), \ldots, (symbol_n, type_n))} $$
$$ t_1 \otimes t_2 \otimes \ldots \otimes t_n $$

$\mathit{arity}(T) = \max_{i=1}^{n}{arity(t_i)}$\\
$\mathit{known\_size}(T) = \left\land_{i=1}^{n} \mathit{known\_size(t_i)}$\\

%% Re. operations: mention export trick (exported field ref splitter returns record of access rights to each relevant field; with suitable (guaranteed?) representation optimations, this virtual record gives no overhead.)

\subsection{Sum types (variants)}
$$ \mathit{sum((symbol_1, type_1), \ldots, (symbol_n, type_n))} $$
$$ t_1 \oplus t_2 \oplus \ldots \oplus t_n $$

$\mathit{arity}(T) = \max_{i=1}^{n}{arity(t_i)}$\\
$\mathit{known\_size}(T) = \left\land_{i=1}^{n} \mathit{known\_size(t_i)}$\\

\subsection{Vtable types}
$$ \mathit{vtable(parent, sig_1, \ldots, sig_n)} $$
$$ sig_1! \otimes sig_2! \otimes \ldots \otimes sig_n! $$
%% TODO: Where to put the '!' ? Prefix or suffix?

$\mathit{arity}(T) = \max(\mathit{arity}(parent), \max_{i=1}^{n}{arity(sig_i)})$\\
$\mathit{known\_size}(T) = \top$\\

\subsection{Pointer types}

$\mathit{known\_size}(T) = \top$\\

\subsection{Type variables}
Type variables are encoded using de Bruijn indices.
$$ \mathit{var(i)} $$

$\mathit{arity}(T) = i$\\
$\mathit{known\_size}(T) = \bot$\\

\subsection{Type application}
$$ \mathit{apply(type_{abs},type_{arg})} $$

$\mathit{arity}(T) = \max(\mathit{arity}(type_{abs}) - 1, \mathit{arity}(type_{arg}))$

\subsection{Existential types}
%  TODO: Add size of existential parameter
$$ \mathit{\exists(type)} $$

$\mathit{arity}(T) = \mathit{arity}(type) - 1$

\subsection{Recursive types (letrec)}
$$ \mathit{letrec(type_1, \ldots, type_n)} $$

$\mathit{arity}(T) = (\max_{i=1}^{n}{arity(type_i)}) - n$

\subsection{Symbolic types (tags)}

\paragraph{References} are a special kind of symbolic tag
named \texttt{linearvm.basis.ref}.


\subsection{Special types}

\paragraph{Never} (aka bottom) -- signifies ``can't return'' and is
recognized and treated specially by the validator.
Is equivalent to the empty variant.

\paragraph{Void} (aka unit) -- the type containing exactly one value.
Is equivalent to the empty record.

\section{Type rules}

\section{Type-specific operations}
% Auto-rules ("axiom schemas") for product/sum/vtable/... types.





\chapter{File format}
\label{ch:fileformat}

\section{File encoding}

\section{File sections}
% TOC: (<section-id:32> <start-offset:32>)* ; section-id=0 markerer slut.

\begin{tabular}{ll}
\hline
\hline
Identifier & Section contents\\
\hline
\tt SymS & Symbol source\\        %
\tt Symb & Symbols\\              % -> SSrc
\tt Name & Names\\                % -> Symb

\tt TImp & Imported types\\       % -> Name
\tt FImp & Imported functions\\   % -> FSig

\tt Type & Type definitions\\     % -> TImp + Symb
\tt FSig & Function signatures\\  % -> Type
\tt FunD & Function declarations\\ % -> FSig + Symb + Cnst
% TODO: VTables
\tt CstS & Constant pool source\\ %
\tt CDec & Constant decoders\\    % -> FImp + ?
\tt Cnst & Constants\\            % -> CBin+CDec

\tt Func & Function definitions\\ % -> FSig + Symb + Cnst
\tt TExp & Exported types\\       % -> Symb + Type
\tt FExp & Exported functions\\   % -> Symb + FSig
\hline
\hline
\end{tabular}
% TODO: Function declarations vs. definitions -- How to represent it?
% Logically,
% Function declaration -> name + signature + provider + details


\subsection{Symbol source}
The Symbol Source section is a binary section, consisting of UTF-8 encoded unicode characters.

\paragraph{Requirements:}
\begin{enumerate}
\item The entire section is a sequence of valid UTF-8-encoded characters.
\end{enumerate}

\subsection{Symbols}
The Symbols section is a number-sequence section. It defines a set of
symbols, numbered $1 \ldots n_{symbols}$.
The following $n_{symbols}$ numbers are the lengths, in bytes, of the symbols.
\begin{quote}
  \texttt{Symb} ::= <$n_{symbols}$> <$symbol\_length$>[1 \ldots $n_{symbols}$]
\end{quote}

\paragraph{Interpretation:}
The $i$th symbol is defined as the substring of the Symbols Source section
which begins at offset $\sum_{j=1}^{i-1} {symbol\_length_{j}}$
and ends at offset $\sum_{j=1}^{i} {symbol\_length_{j}}$.

\paragraph{Requirements:}
\begin{enumerate}
\item The section contains exactly $1 + n_{symbols}$ numbers.
\item The symbol lengths sum up to the length (in bytes) of the Symbol Source section.
\item The $i$th symbol is a valid UTF-8 string.
\end{enumerate}

\subsection{Names}
The Names section is a number-sequence section.
It defines a set of names, numbered $1 \ldots n_{names}$.
The following $2 \cdot n_{names}$ numbers are pairs $(parent\_name, symbol)$ of indices.

\begin{quote}
  \texttt{Symb} ::= <$n_{names}$> <$name$>[1 \ldots $n_{names}$]\\
  \texttt{<name>} ::= <$parent\_name$> <$symbol$>
\end{quote}

\paragraph{Interpretation:}
Names are sequences of symbols, conventionally written with periods as
separators. The names form a hierarchy: names with $parent\_name = 0$
are roots, consisting only of the single symbol $symbol$; names with
$parent\_name > 0$ are interpreted as the symbol sequence of the
parent name followed by the $symbol$.

\paragraph{Requirements:}
\begin{enumerate}
\item The section contains exactly $1 + 2 \cdot n_{names}$ numbers.
\item For all $i \in [1;n_{names}]$: $0 \le name_i.parent\_name < i$.
\item For all $i \in [1;n_{names}]$: $1 \le name_i.symbol \le n_{symbols}$.
\end{enumerate}



\chapter{Misc}

\section{Suggested representation optimizations}

\paragraph{Records (product types).}
\begin{itemize}
\item Record fields of a virtual type (size 0) are not represented at all.
\item Records consisting solely of fields of a virtual type are not represented at all.
\end{itemize}

\paragraph{Variants (sum types).}
\begin{itemize}
\item Variant types of which no variants are data bearing (i.e., pure enumerations) are represented as just the variant number.
\item Variant types of which exactly one variant is data bearing (of type T), and which have at most 256 other variants, are represented as either a pointer to type T, or as an integer in $[0;255]$.
\item Variant types of which at most 4 variants are data bearing (of type $T_0\ldots T_3$), and which have at most 256 other variants, are represented as either a pointer to type T, tagged in the lowest two bits with $i$, or as an integer in $[0;255]$.
\end{itemize}


\end{document}
