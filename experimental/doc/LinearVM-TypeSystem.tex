\documentclass[a4paper]{report}

\newcommand\arity{{\ensuremath\textit{arity}}}
\newcommand\fixedsize{{\ensuremath\textit{fixed\_size}}}

%\newcommand\max{{\ensuremath\textit{max}}}
\newcommand\implies{\;\Longrightarrow\;}

% Type constructors:
\newcommand\tyImport{{\ensuremath\textit{import}}}
\newcommand\tyVar{{\ensuremath\textit{var}}}
\newcommand\tyApp{{\ensuremath\textit{app}}}
\newcommand\tyBless{{\ensuremath\textit{bless}}}
\newcommand\tyCurse{{\ensuremath\textit{curse}}}
\newcommand\tyPtr{{\ensuremath\textit{ptr}}}
\newcommand\tyPad{{\ensuremath\textit{pad}}}
\newcommand\tyRecord{{\ensuremath\textit{record}}}
\newcommand\tyVariant{{\ensuremath\textit{variant}}}

\newcommand\tyVoid{{\ensuremath\textit{void}}}

\begin{document}

\chapter{Type system}
\label{chap:typesystem}

\section{introduction}

The main property of values and variables in LinearVM are their type.
Any value in LinearVM belongs to a type -- and sometimes to more than
one type.

Most primitive types in LinearVM, in particular the numeric types, are
not listed or described in the type system proper; they are instead
treated as imported types.

All composite types, however, are constructed using the set of
\emph{type constructors} of the type system.

The type constructors belong roughly to the following groups:
\begin{description}
\item[Importing] For linking, and for access to primitive types.
\item[Product and sum types] The bread and butter, or structural
  backbone, of type construction.
\item[Genericity] Parametric polymorphism; type variables and their
  supporting constructs.
\item[Dynamic dispatch] Behavioural polymorphism.
\item[Borrowing] Accessing object fields without having to deconstruct
  the objects.
\item[Tagging] A mechanism which supports creating custom sub- and
  supertypes of any given type, and also describing certain relations
  between different values.
\item[Memory] Pointers and padding.
\end{description}


\section{Overview}

\paragraph{Importing:}
A module (compilation unit) may \emph{import} types from other modules.
Built-in types are accessed this way.
In particular, all numeric primitive types are imported; for instance,
the type of 32-bit unsigned integers are imported as
\texttt{linearvm.basis.uint}, and double-precision floating point
numbers are importeded as \texttt{linearvm.basis.double}.

\paragraph{Structure:}
Structure is described using product and sum types. In LinearVM, they
are known as \emph{record} and \emph{variant} types.
Enumeration types are a special case of variant types.
Booleans, for example, are defined as a variant type with two cases,
each carrying a unit (void) type.
The unit type (called \emph{void}) is simply the record type with zero fields.

\paragraph{Genericity:}
Genericity, or parametric polymorphism, is described using
\emph{type variables} and \emph{type application}.
For instance, a pair of $T$s for any type $T$ can be defined as
$$pair[T] := \forall T:record\{a:T, B:T\}$$
and instantiated like
$$pair[bool]$$
(in reality, the $\forall$ quantor is implied rather than explicit).

\paragraph{Recursive types:}
Recursive data types also use type variables.
For instance, a cons list of \texttt{int32}s must be defined as a recursive type,
which can be done like this:
$$
int32list := rec(T, variant\{nil:void, cons:record\{head:\mathtt{int32}, tail:ptr(T)\}\}
$$
Note that this definition uses a pointer at the point of
recursion. This is necessary for the type to have a finite size. More
about this later.

The recursion type constructor can be used on several type variables at the same time, to define mutually recursive types.

\paragraph{Dynamic dispatch:}
Different programming paradigms have different names for their mechanism
for behavioural polymorphism: function pointer, closure, vtable.
In LinearVM it's called a \emph{vtable}.
A vtable is a record of function pointers. Vtables are static entites, and
can only be constructed at compile time, not at runtime.

\paragraph{Borrowing:}
In a purely linear type system, you can't access fields within a
structure while they are part of the structure. This means that in
order to read e.g.  a numeric value out of a structure, you must
decompose the structure and then put it back together afterwards.
This is clearly not satisfactory for practical use.

To overcome this limitation, we have \emph{borrowing} -- types which
specify that this is not an owning reference, but merely borrowing
with the obligation to return the value again (release the borrowing),
while being ensured that the real owner is not allowed to do
modifications to the value such as disposing of it) while it is lend
out.

% TODO: shared vs. exclusive; stack-to-stack borrowing vs. general case.

\paragraph{Tagging:}
% TODO -- incl. encapsulation

\paragraph{Memory:}
% TODO -- pointers (w/ design choice); padding


\subsection{Typical combinations}
\begin{itemize}
\item Pointers and type recursion -- for recursive data types.
\item Vtables and existential types (and pointers) -- for closures.
\item Existential types and padding.
\end{itemize}



\section{Type properties}

The type rules make use of certain type properties associated with each type.
These properties are described briefly here and defined formally later,
as we encounter the type constructors.

\subsection{Arity}
Higher-order types are supported. To each type is therefore associated
an \emph{arity} property which tells the depth of its free variables.
$$
\arity(T) : int
$$
In the binary file format, type variables are not named, but specified
by their de Bruijn indices.

\subsection{Assignability}
Central to the type rules are the restrictions of value movements,
which are governed by \emph{assignability}:
$$
T_{\it destination} := T_{\it source}
$$

Assignability is reflexive:
$$T := T$$

\subsection{Fixed-size}
While polymorphic types are supported, the representation size of values
which may inhabit registers must always be known at runtime (and
determinable at load time).
The type property \emph{fixed\_size} tells whether this is the case.
$$
\fixedsize(T)
$$

It is always the case that
$$
\arity(T) = 0 \implies \fixedsize(T)
$$

\section{Type constructors}

\subsection{Imported types}
A type occurring in a module (compilation unit) can be imported from
outside the module.
At the import point, an arity and \emph{view type} must be specified for it.
The same is true of the point of export.
$$
\it type ::= \tyImport(name : qsymbol, arity : int, view\_type : type\_or\_opaque)
$$
At load time, it is verified that
\begin{itemize}
\item the imported type exists, exported with $\arity_{\it export}$
  and $view\_type_{\it export}$
\item the arities match: $\arity_{\it export} = \arity_{\it import}$
\item the view types match:\\
$view\_type_{\it import} = view\_type_{\it export} \;\lor\; view\_type_{\it import} = opaque$
\item the view type matches the arity: $\arity(view\_type_{\it import}) \le \arity_{\it import}$
% TODO: This might be generalized. These special cases are safe.
\end{itemize}

\subsubsection*{Properties}
$$\arity(\tyImport(n, a, t)) = a$$

$$
{t \neq opaque
\over
\tyImport(n,a,t) := t}
\quad
{t \neq opaque
\over
t := \tyImport(n,a,t)}
$$

$$
{a=0 \over \fixedsize(\tyImport(n, a, t))}
\quad
{\fixedsize(t) \over \fixedsize(\tyImport(n, a, t))}
$$


\subsection{Type variables}
Type variables are specified by their de Bruijn indices.
$$
\it type ::= \tyVar(i : int)
$$

\subsubsection*{Properties}
$$\arity(\tyVar(i)) = i$$

\subsection{Type application}
$$
\it type ::= \tyApp(type : type, arg : type)
$$
\subsubsection*{Properties}
$$ \arity(\tyApp(t,a)) = \max(\arity(t)-1, \arity(a)) $$

\subsection{Recursive type application}
\subsection{Universal quantification}
% Universal quantification is often implicit:
\subsection{Existential quantification}
Existential quantification binds a type variable existentially.
$$ \it type ::= \exists(type : type) $$

\subsubsection*{Properties}
$$ \arity(\exists(t)) = \arity(t)-1 $$

$${s \textrm{ is any type}} \over \exists(t) := t(s)$$

$$\fixedsize(t) \over \fixedsize(\exists(t))$$

\subsection{Tagging}
\emph{Tagging} is a type encapsulation/subtyping mechanism.
It comes in two flavours: \emph{cursing} and \emph{blessing}.

\subsubsection{Cursing}
\emph{Cursing} is a type encapsulation/subtyping mechanism.
Covariantly, it is a way of marking that a type is restricted in its use.
Contravariantly, it is a way of marking that a given restriction does not apply.
$$
\it type ::= \tyCurse(name : qsymbol, args : type\_tag\_arg^*, type : type)
$$
$$
\it type\_tag\_arg ::= symbol \;|\; integer \;|\; type
$$

\subsubsection*{Properties}
$$
\arity(\tyCurse(n,a,t)) = \max(\arity(t), \max_{t' \in a} \arity(t'))
$$
$$
\tyCurse(n,a,t) := t
$$

$$\fixedsize(t) \over \tyCurse(n,a,t)$$

\subsubsection{Blessing}
\emph{Blessing} is a subtyping mechanism.
Covariantly, it is a way of marking that a type can be used in an
extended set of circumstances.
Contravariantly, it is a way of marking that a given blessing is required.

$$
\it type ::= \tyBless(name : qsymbol, args : type\_tag\_arg^*, type : type)
$$

\subsubsection*{Properties}
$$
\arity(\tyBless(n,a,t)) = \max(\arity(t), \max_{t' \in a} \arity(t'))
$$
$$
t := \tyBless(n,a,t)
$$

$$\fixedsize(t) \over \fixedsize(\tyBless(n,a,t))$$

\subsection{Borrowing}


\subsection{Pointers}
The purpose of pointer types is to represent data structures of variable length
in a fixed-size field.
Pointers have no ``null''-like value. Pointer types should in practise
always be used in association with a specification of a memory management
policy.

$$\it type ::= \tyPtr(type : type)$$

\subsubsection*{Properties}
$$ \arity(\tyPtr(t)) = \arity(t)$$

All pointer types are fixed-size.
$$\fixedsize(\tyPtr(t))$$

\subsection{Padding}
At times it is necessary or convenient to ensure that a field's size is
(physically) large enough to accommodate all of a set of types.

$$\it type ::= \tyPad(type : type, pad\_type : type)$$

Padding is used in conjunction with pointers, existential types,
and types with lifecycles.

\subsubsection*{Remarks}
An $n$-ary version of $\tyPad$ can be had through nesting:
$$\tyPad(t, \tyPad(t_{pad1}, t_{pad2}))$$

\noindent
% For all intents and purposes, ...
Intuitively, the following equivalenses hold:

\begin{itemize}
\item Commutativity of padding types:
$$\tyPad(t, \tyPad(t_{pad1}, t_{pad2})) \equiv
  \tyPad(t, \tyPad(t_{pad2}, t_{pad1}))$$

\item Associativity of padding types:
$$\tyPad(t, \tyPad(t_{pad1}, \tyPad(t_{pad2}, t_{pad3}))) \equiv
\tyPad(t, \tyPad(t_{pad2}, \tyPad(t_{pad1}, t_{pad3})))$$

\item $\tyVoid$ as a neutral element:
$$\tyPad(t, \tyVoid) \equiv t$$

\end{itemize}

\subsubsection*{Properties}
$$ \arity(\tyPad(t, t')) = \max(\arity(t), \arity(t'))$$


$${\fixedsize(t) \qquad \fixedsize(t')}
 \over
\fixedsize(\tyPad(t, t'))$$

\subsection{Records}
Record types describe conjunctions of types -- the simultaneous
existence of multiple values (of similar or different types).

$$\it type ::= \tyRecord(fields : field^*)$$
$$\it field ::= \{name:symbol, type:type\}$$

The field names must be distinct within a record type.

\subsubsection*{Properties}

$$
\arity(\tyRecord({\it fields})) = \max_{f \in {\it fields}}(\arity(f.{\it type}))
$$


A record type is fixed-size when all of its fields are.
$${\forall f \in {\it fields} : \fixedsize(f.{\it type})} \over
  \fixedsize(\tyRecord({\it fields}))
$$

\subsection{Variants}
Variant types describe disjunctions of types -- the
existence of a value of exactly one of multiple (similar or different) types.

$$\it type ::= \tyVar(flavours : flavour^*)$$
$$\it flavour ::= \{name:symbol, type:type\}$$

The flavour names must be distinct within a variant type.

\subsubsection*{Properties}

$$
\arity(\tyVar({\it flavours})) = \max_{f \in {\it flavours}}(\arity(f.{\it type}))
$$


A variant type is fixed-size when all of its fields are.
$${\forall f \in {\it flavours} : \fixedsize(f.{\it type})} \over
  \fixedsize(\tyVar({\it flavours}))
$$

\end{document}
