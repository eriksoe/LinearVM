\documentclass[a4paper]{report}

\newcommand\arity{{\ensuremath\textit{arity}}}
\newcommand\fixedsize{{\ensuremath\textit{fixed\_size}}}

% Type constructors:
\newcommand\tyImport{{\ensuremath\textit{import}}}
\newcommand\tyVar{{\ensuremath\textit{var}}}
\newcommand\tyApp{{\ensuremath\textit{app}}}
\newcommand\tyBless{{\ensuremath\textit{bless}}}
\newcommand\tyCurse{{\ensuremath\textit{curse}}}
\newcommand\tyPtr{{\ensuremath\textit{ptr}}}

\begin{document}

\chapter{Type system}
\label{chap:typesystem}

\section{Type properties}

The type rules make use of certain type properties associated with each type.
These properties are described briefly here and defined formally later,
as we encounter the type constructors.

\subsection{Arity}
Higher-order types are supported. To each type is therefore associated
an \emph{arity} property which tells the depth of its free variables.
$$
\arity(T) : int
$$
Type variables are not named, but specified by their de Bruijn indices.

\subsection{Assignability}
Central to the type rules are the restrictions of value movements,
which are governed by \emph{assignability}:
$$
T_{\it destination} := T_{\it sourceff}
$$

Assignability is reflexive:
$$T := T$$c

\subsection{Fixed-size}
While polymorphic types are supported, the representation size of values
which may inhabit registers must always be known at runtime (and
determinable at load time).
The type property \emph{fixed\_size} tells whether this is the case.
$$
\fixedsize(T)
$$

\section{Type constructors}

\subsection{Imported types}
A type occurring in a module (compilation unit) can be imported from
outside the module.
At the import point, an arity and \emph{view type} must be specified for it.
The same is true of the point of export.
$$
\it type ::= \tyImport(name : qsymbol, arity : int, view\_type : type\_or\_opaque)
$$
At load time, it is verified that
\begin{itemize}
\item the arities match: $\arity_{\it export} = \arity_{\it import}$
\item the view types match:\\
$view\_type_{\it export} = view\_type_{\it import} \;\lor\; view\_type_{\it import} = opaque$
% TODO: This might be generalized. These special cases are safe.
\end{itemize}

\subsubsection*{Properties}
$$\arity(\tyImport(n, a, t)) = a$$

$$
{t \neq opaque
\over
\tyImport(n,a,t) := t}
\quad
{t \neq opaque
\over
t := \tyImport(n,a,t)}
$$

$$
{a=0 \over \fixedsize(\tyImport(n, a, t))}
\quad
{\fixedsize(t) \over \fixedsize(\tyImport(n, a, t))}
$$


\subsection{Type variables}
Type variables are specified by their de Bruijn indices.
$$
\it type ::= \tyVar(i : int)
$$

\subsubsection*{Properties}
$$\arity(\tyVar(i)) = i$$

\subsection{Type application}
$$
\it type ::= \tyApp(type : type, arg : type)
$$
\subsubsection*{Properties}
$$ \arity(\tyApp(t,a)) = max(\arity(t)-1, \arity(a)) $$

\subsection{Recursive type application}
\subsection{Universal quantification}
% Universal quantification is often implicit:
\subsection{Existential quantification}
Existential quantification binds a type variable existentially.
$$ \it type ::= \exists(type : type) $$

\subsubsection*{Properties}
$$ \arity(\exists(t)) = \arity(t)-1 $$

$${s \textrm{ is any type}} \over \exists(t) := t(s)$$


\subsection{Cursing}
\emph{Cursing} is a type encapsulation/subtyping mechanism.
It is a way of marking that a type is restricted in its use.
$$
\it type ::= \tyCurse(name : qsymbol, args : type\_tag\_arg^*, type : type)
$$
$$
\it type\_tag\_arg ::= symbol \;|\; integer \;|\; type
$$

\subsubsection*{Properties}
$$
\tyCurse(n,a,t) := t
$$

\subsection{Blessing}
\emph{Blessing} is a subtyping mechanism.
It is a way of marking that a type can be used in an extended set of
circumstances.

$$
\it type ::= \tyBless(name : qsymbol, args : type\_tag\_arg^*, type : type)
$$

\subsubsection*{Properties}
$$
t := \tyBless(n,a,t)
$$

\subsection{Pointers}
The purpose of pointer types is to represent data structures of variable length
in a fixed-length field.
Pointers have no ``null''-like value. Pointer types should in practise
always be used in association with a particular memory management
policy.

$$\it type ::= \tyPtr(type : type)$$

\subsubsection*{Properties}
$$ \arity(ptr(t)) = \arity(t)$$

All pointer types are fixed-size.
$$\fixedsize(ptr(t))$$

\end{document}
