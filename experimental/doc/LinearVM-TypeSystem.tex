\documentclass[a4paper]{report}

\newcommand\arity{{\ensuremath\textit{arity}}}
\newcommand\fixedsize{{\ensuremath\textit{fixed\_size}}}

%\newcommand\max{{\ensuremath\textit{max}}}
\newcommand\implies{\;\Longrightarrow\;}

% Type constructors:
\newcommand\tyImport{{\ensuremath\textit{import}}}
\newcommand\tyVar{{\ensuremath\textit{var}}}
\newcommand\tyApp{{\ensuremath\textit{app}}}
\newcommand\tyBless{{\ensuremath\textit{bless}}}
\newcommand\tyCurse{{\ensuremath\textit{curse}}}
\newcommand\tyPtr{{\ensuremath\textit{ptr}}}
\newcommand\tyPad{{\ensuremath\textit{pad}}}
\newcommand\tyRecord{{\ensuremath\textit{record}}}
\newcommand\tyVariant{{\ensuremath\textit{variant}}}

\newcommand\tyVoid{{\ensuremath\textit{void}}}

\begin{document}

\chapter{Type system}
\label{chap:typesystem}

\section{Type properties}

The type rules make use of certain type properties associated with each type.
These properties are described briefly here and defined formally later,
as we encounter the type constructors.

\subsection{Arity}
Higher-order types are supported. To each type is therefore associated
an \emph{arity} property which tells the depth of its free variables.
$$
\arity(T) : int
$$
In the binary file format, type variables are not named, but specified
by their de Bruijn indices.

\subsection{Assignability}
Central to the type rules are the restrictions of value movements,
which are governed by \emph{assignability}:
$$
T_{\it destination} := T_{\it source}
$$

Assignability is reflexive:
$$T := T$$

\subsection{Fixed-size}
While polymorphic types are supported, the representation size of values
which may inhabit registers must always be known at runtime (and
determinable at load time).
The type property \emph{fixed\_size} tells whether this is the case.
$$
\fixedsize(T)
$$

It is always the case that
$$
\arity(T) = 0 \implies \fixedsize(T)
$$

\section{Type constructors}

\subsection{Imported types}
A type occurring in a module (compilation unit) can be imported from
outside the module.
At the import point, an arity and \emph{view type} must be specified for it.
The same is true of the point of export.
$$
\it type ::= \tyImport(name : qsymbol, arity : int, view\_type : type\_or\_opaque)
$$
At load time, it is verified that
\begin{itemize}
\item the imported type exists, exported with $\arity_{\it export}$
  and $view\_type_{\it export}$
\item the arities match: $\arity_{\it export} = \arity_{\it import}$
\item the view types match:\\
$view\_type_{\it import} = view\_type_{\it export} \;\lor\; view\_type_{\it import} = opaque$
\item the view type matches the arity: $\arity(view\_type_{\it import}) \le \arity_{\it import}$
% TODO: This might be generalized. These special cases are safe.
\end{itemize}

\subsubsection*{Properties}
$$\arity(\tyImport(n, a, t)) = a$$

$$
{t \neq opaque
\over
\tyImport(n,a,t) := t}
\quad
{t \neq opaque
\over
t := \tyImport(n,a,t)}
$$

$$
{a=0 \over \fixedsize(\tyImport(n, a, t))}
\quad
{\fixedsize(t) \over \fixedsize(\tyImport(n, a, t))}
$$


\subsection{Type variables}
Type variables are specified by their de Bruijn indices.
$$
\it type ::= \tyVar(i : int)
$$

\subsubsection*{Properties}
$$\arity(\tyVar(i)) = i$$

\subsection{Type application}
$$
\it type ::= \tyApp(type : type, arg : type)
$$
\subsubsection*{Properties}
$$ \arity(\tyApp(t,a)) = \max(\arity(t)-1, \arity(a)) $$

\subsection{Recursive type application}
\subsection{Universal quantification}
% Universal quantification is often implicit:
\subsection{Existential quantification}
Existential quantification binds a type variable existentially.
$$ \it type ::= \exists(type : type) $$

\subsubsection*{Properties}
$$ \arity(\exists(t)) = \arity(t)-1 $$

$${s \textrm{ is any type}} \over \exists(t) := t(s)$$

$$\fixedsize(t) \over \fixedsize(\exists(t))$$

\subsection{Tagging}
\emph{Tagging} is a type encapsulation/subtyping mechanism.
It comes in two flavours: \emph{cursing} and \emph{blessing}.

\subsubsection{Cursing}
\emph{Cursing} is a type encapsulation/subtyping mechanism.
Covariantly, it is a way of marking that a type is restricted in its use.
Contravariantly, it is a way of marking that a given restriction does not apply.
$$
\it type ::= \tyCurse(name : qsymbol, args : type\_tag\_arg^*, type : type)
$$
$$
\it type\_tag\_arg ::= symbol \;|\; integer \;|\; type
$$

\subsubsection*{Properties}
$$
\arity(\tyCurse(n,a,t)) = \max(\arity(t), \max_{t' \in a} \arity(t'))
$$
$$
\tyCurse(n,a,t) := t
$$

$$\fixedsize(t) \over \tyCurse(n,a,t)$$

\subsubsection{Blessing}
\emph{Blessing} is a subtyping mechanism.
Covariantly, it is a way of marking that a type can be used in an
extended set of circumstances.
Contravariantly, it is a way of marking that a given blessing is required.

$$
\it type ::= \tyBless(name : qsymbol, args : type\_tag\_arg^*, type : type)
$$

\subsubsection*{Properties}
$$
\arity(\tyBless(n,a,t)) = \max(\arity(t), \max_{t' \in a} \arity(t'))
$$
$$
t := \tyBless(n,a,t)
$$

$$\fixedsize(t) \over \fixedsize(\tyBless(n,a,t))$$

\subsection{Pointers}
The purpose of pointer types is to represent data structures of variable length
in a fixed-size field.
Pointers have no ``null''-like value. Pointer types should in practise
always be used in association with a specification of a memory management
policy.

$$\it type ::= \tyPtr(type : type)$$

\subsubsection*{Properties}
$$ \arity(\tyPtr(t)) = \arity(t)$$

All pointer types are fixed-size.
$$\fixedsize(\tyPtr(t))$$

\subsection{Padding}
At times it is necessary or convenient to ensure that a field's size is
(physically) large enough to accommodate all of a set of types.

$$\it type ::= \tyPad(type : type, pad\_type : type)$$

Padding is used in conjunction with pointers, existential types,
and types with lifecycles.

\subsubsection*{Remarks}
An $n$-ary version of $\tyPad$ can be had through nesting:
$$\tyPad(t, \tyPad(t_{pad1}, t_{pad2}))$$

\noindent
% For all intents and purposes, ...
Intuitively, the following equivalenses hold:

\begin{itemize}
\item Commutativity of padding types:
$$\tyPad(t, \tyPad(t_{pad1}, t_{pad2})) \equiv
  \tyPad(t, \tyPad(t_{pad2}, t_{pad1}))$$

\item Associativity of padding types:
$$\tyPad(t, \tyPad(t_{pad1}, \tyPad(t_{pad2}, t_{pad3}))) \equiv
\tyPad(t, \tyPad(t_{pad2}, \tyPad(t_{pad1}, t_{pad3})))$$

\item $\tyVoid$ as a neutral element:
$$\tyPad(t, \tyVoid) \equiv t$$

\end{itemize}

\subsubsection*{Properties}
$$ \arity(\tyPad(t, t')) = \max(\arity(t), \arity(t'))$$


$${\fixedsize(t) \qquad \fixedsize(t')}
 \over
\fixedsize(\tyPad(t, t'))$$

\subsection{Records}
Record types describe conjunctions of types -- the simultaneous
existence of multiple values (of similar or different types).

$$\it type ::= \tyRecord(fields : field^*)$$
$$\it field ::= \{name:symbol, type:type\}$$

The field names must be distinct.

\subsubsection*{Properties}

$$
\arity(\tyRecord({\it fields})) = \max_{f \in {\it fields}}(\arity(f.{\it type}))
$$


A record type is fixed-size when all of its fields are.
$${\forall f \in {\it fields} : \fixedsize(f.{\it type})} \over
  \fixedsize(\tyRecord({\it fields}))
$$

\subsection{Variants}
\subsubsection*{Properties}

\end{document}
